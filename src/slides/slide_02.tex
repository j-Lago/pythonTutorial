%! Author = j-Lago
%! Date = 5/19/2025

\begin{frame}{Slide 2}

Exemplo de gráfico utilizando \texttt{tikz}:
\begin{center}
    \begin{tikzpicture}
        \begin{axis}[
            width=280px,
            height=160px,
            axis lines = middle,
            enlargelimits = true,
            xlabel = $x$,
            ylabel = $y$,
            legend pos = north east,
            grid = major,
            legend style={fill=background, draw=textcolor, text=textcolor},
            yticklabel style={xshift=-20pt},
            axis line style={thick, textcolor},
            grid style={solid, textcolor!30!background, line width=0.2pt}
        ]
            \addplot [primary, thick, domain=0:10, samples=100]
                {exp(-0.4*x) * sin(deg(x))};
            \addlegendentry{$e^{-0.4x} \sin(x)$}

            \addplot [secondary, thick, dashed, domain=0:10, samples=100]
                {exp(-0.2*x) * cos(deg(x))};
            \addlegendentry{$e^{-0.2x} \cos(x)$}

%            \draw[thick, white] (axis cs:0,\pgfkeysvalueof{/pgfplots/ymin}) -- (axis cs:0,\pgfkeysvalueof{/pgfplots/ymax});
%            \draw[thick, white] (axis cs:\pgfkeysvalueof{/pgfplots/xmin},0) -- (axis cs:\pgfkeysvalueof{/pgfplots/xmax},0);

        \end{axis}
    \end{tikzpicture}
\end{center}

Exemplo de equação:
    $$
    f(t) = \sum_{h=1,3\ldots}^\infty \dfrac{1}{h!} \sin(h \omega t)
    $$



\end{frame}
