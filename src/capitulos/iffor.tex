\chapter{Estruturas de controle de fluxo}\label{iffor}

Até agora, os pequenos trechos de código que escrevemos, majoritariamente, seguiram uma sequência linear e
bem definida, executando instruções em uma ordem preestabelecida, onde cada linha era processada
incondicionalmente --- como seguir uma receita de bolo à risca.

No entanto, essa abordagem não é suficiente para representar comportamentos mais complexos, onde o próprio algoritmo
precisa tomar decisões com base no estado atual.
Por exemplo, em determinado ponto da execução, se uma condição específica for verdadeira, o programa pode seguir
pelo caminho \inlcode{A}; caso contrário, deve seguir pelo caminho \inlcode{B}.
Cada um desses caminhos corresponde a blocos de código distintos, que realizam ações diferentes para atingir um
certo objetivo.


Além das decisões condicionais, os programas frequentemente precisam executar uma mesma ação várias vezes, sem que o
programador precise escrever o mesmo código repetidamente.
Além de ser impraticável, muitas vezes a lógica desejada exige que a decisão sobre quantas vezes executar uma tarefa
seja tomada durante a própria execução do programa, baseada no estado atual.

Para isso, utilizamos as estruturas de repetição (ou loops), que permitem a execução contínua de um bloco de código
enquanto certas condições forem atendidas, ou garantem que um bloco de código seja executado para todos os valores
de uma lista, de forma iterativa.
Dessa forma, o programa pode automatizar tarefas, tornando o código mais eficiente e adaptável.

Em qualquer programa de computador, controlar a ordem de execução das instruções é fundamental para garantir que a
aplicação possa lidar com diferentes cenários, tomar decisões dinamicamente e repetir ações conforme necessário.
Esse controle é realizado por meio das estruturas de controle de fluxo, que possibilitam modificar o comportamento
de um programa baseado no esta do atual, tornando-o mais flexível e inteligente.

Como aludido anteriormente, as estruturas de controle de fluxo podem ser classificadas em duas categorias: estruturas
condicionais e estruturas de repetição.
Sua implementação pode variar entre diferentes linguagens, e até mesmo dentro
de uma mesma linguagem, podem existir múltiplas variantes para tornar seu uso mais conveniente em diferentes situações.

Por exemplo, em Python, decisões condicionais podem ser feitas com \inlcode{if-else}, uma abordagem mais direta, ou
com \inlcode{match-case}, que facilita comparações mais organizadas entre múltiplas possibilidades.
Da mesma forma, repetições podem ser controladas com \inlcode{while}, quando depende de uma condição, ou
com \inlcode{for}, que é ideal para percorrer sequências.

Essa flexibilidade permite que programadores escolham a estrutura mais adequada para cada contexto, tornando o
código mais simples, expressivo e eficiente.

Nas próximas seções, exploraremos cada categoria detalhadamente, com exemplos para ilustrar seu funcionamento.



\section{Estruturas condicionais}\label{if}
As estruturas condicionais permitem que um programa tome decisões durante sua execução com base em condições
estabelecidas.
Em Python, existem diferentes variantes dessa estrutura, tornando-a mais versátil para diversas situações.
A abordagem mais comum é o uso de \inlcode{if-else}, onde um bloco de código é executado apenas se uma condição for
verdadeira.
Para comparações múltiplas podes usar \inlcode{if-elif-else} ou \inlcode{match-case},
introduzido no Python 3.10, oferece uma alternativa estruturada, útil quando há diversas possibilidades de decisão.


\subsection{\inlcode{if-else}}
Exemplo do uso de \inlcode{if-else}:
\begin{minted}{custompython}
idade = 20

if idade >= 18:
    print("Você é maior de idade.")  # é executado se idade for maior ou igual a 18
else:
    print("Você é menor de idade.")  # é executado se idade menor que 18
\end{minted}

O comando \inlcode{if} sempre espera uma expressão que resulte em um valor booleano,
ou seja, \inlcode{True} (verdadeiro) ou \inlcode{False} (falso).

Quando a condição for avaliada como \inlcode{True}, o bloco de código associado ao \inlcode{if} será executado.
Caso contrário, se houver um \inlcode{else} (que é opcional), seu respectivo bloco será executado.

Cada bloco pode ter um tamanho arbitrário e conter múltiplas instruções, incluindo estruturas \inlcode{if-else}
aninhadas, permitindo expressar deci~soes mais complexas.
Assim como nas funções, a delimitação de um bloco \inlcode{if} é feita pela indentação.

No exemplo, dependendo do valor da variável idade a comparação \inlcode{idade >= 18} retorna \inlcode{True} ou
\inlcode{False}, e baseado nesse valor o \inlcode{if} escolhe qual bloco de código executar, exibindo a mensagem apropriada.

Essas expressões booleanas geralmente envolvem o uso de operadores relacionais
(\inlcode{==}, \inlcode{!=}, \inlcode{>}, \inlcode{<}, \inlcode{>=}, \inlcode{<=}), que podem ser combinados com
operadores lógicos (\inlcode{and}, \inlcode{or}, \inlcode{not}) para formar condições lógicas mais complexas.
Além disso, o operador \inlcode{is} verifica se duas variáveis referenciam o mesmo objeto na memória, diferindo
do \inlcode{==}, que compara os valores desses objetos.
O operador \inlcode{in} é útil para verificar se um determinado valor está presente em uma coleção, como uma lista.
Outra função embutida \inlcode{isinstance(obj, class)} é especialmente útil para verificar o tipo de um objeto
dentro dessas expressões.

\subsection{\inlcode{if-elif-else}}
Segue um exemplo de uma função que verifica o cadastro de um usuário para identificar um meio de contato válido.
A prioridade é o email; caso não esteja disponível, o número de telefone será retornado como alternativa.
\begin{minted}{custompython}
def obter_contato(cadastro: dict) -> str:
    if 'email' in cadastro and cadastro['email'] != '':
        return cadastro['email']
    elif 'telefone' in cadastro and cadastro['telefone'] != '':
        return cadastro['telefone']
    else:
        return 'nenhum contato cadastrado'

usuario1 = {'nome': 'Fulano', 'email': 'fulano@ifsc.edu.br', 'telefone': '(48)99999-9999'}
usuario2 = {'nome': 'Beltrano', 'telefone': '(48)99999-8888'}
usuario3 = {'nome': 'Sicrano', 'telefone': ''}

print('contato usuario1:', obter_contato(usuario1))
print('contato usuario2:', obter_contato(usuario2))
print('contato usuario3:', obter_contato(usuario3))
\end{minted}
\begin{minted}{text}
contato usuario1: fulano@ifsc.edu.br
contato usuario2: (48)99999-8888
contato usuario3: nenhum contato cadastrado
\end{minted}

\subsection{\inlcode{match-case}}
A seguir temos um exemplo usando \inlcode{match-case} para verificar se um determinado dia da semana é um final de
semana ou um dia útil:
\begin{minted}{custompython}
dia_da_semana = "sábado"

match dia_da_semana.lower():
    case "sábado" | "domingo":
        print("É fim de semana! Aproveite para descansar.")
    case "segunda" | "terça" | "quarta" | "quinta" | "sexta":
        print("É dia útil. Hora de trabalhar ou estudar!")
    case _: # executado caso não haja matching
        print("Dia inválido. Certifique-se de inserir um nome correto.")
\end{minted}

Esse código verifica o valor da variável \inlcode{dia_da_semana} e determina se o dia pertence ao final de semana ou
se é um dia útil, comparando-o com as opções especificadas nos diferentes casos (\inlcode{case}) dentro da estrutura
\inlcode{match-case}\footnote{em Python
\inlcode{match-case} é uma estrutura mais completa que não realiza apenas comparações de valores, mas sim
\emph{pattern matching}, que envolve conceitos de desestruturação e correspondência de padrões. Conceitos um pouco mais avançados que ficarão de fora desse tutorial introdutório.}.


Cada \inlcode{case} dentro de um \inlcode{match} representa um possível caso de correspondência, permitindo que o
programa escolha dinamicamente a execução adequada de apenas um deles.

O operador \inlcode{|} possibilita agrupar múltiplos casos numa mesma cláusula, tornando o código mais legível e eficiente.

Além disso, a conversão para minúsculas (\inlcode{lower()}) garante que a entrada do usuário funcione corretamente,
independentemente de letras maiúsculas ou minúsculas.





\section{Estruturas de repetição}\label{for}

As estruturas de repetição permitem que um programa execute um bloco de código múltiplas vezes de forma
controlada, eliminando repetições manuais e tornando o código mais legível e organizado.
Além de aprimorar a clareza, elas aumentam a eficiência e versatilidade do desenvolvimento ao possibilitar
que o número de execuções seja definido dinamicamente durante a execução,
adaptando-se às exigências da lógica que está sendo implementada.

Em Python, assim como na maioria das linguagens, há duas principais formas de implementar repetições:
o comando \inlcode{while}, ideal para execuções dependentes de uma condição dinâmica, e o comando \inlcode{for},
mais adequado para percorrer elementos de uma sequência de maneira previsível.
Cada uma dessas estruturas apresenta vantagens específicas e pode ser aplicada conforme a necessidade do programa,
abrangendo desde a iteração sobre sequências ou coleções de dados até algoritmos que requerem monitoramento
contínuo e definição dinâmica do critério de parada, como cálculos iterativos em modelos computacionais.

\subsection{\inlcode{while}}

O comando \inlcode{while} permite a repetição de um bloco de código enquanto uma condição for verdadeira,
sendo especialmente útil em situações onde o número de iterações não pode ser determinado previamente.
Essa abordagem possibilita a execução de tarefas que dependem de eventos externos,
verificações contínuas ou condições de parada dinâmicas.

O exemplo a seguir apresenta um programa interativo no qual o usuário deve tentar adivinhar um número secreto (\inlcode{57}).
Caso acerte, vence o jogo e o programa é encerrado.
Se errar, poderá continuar tentando até acertar.
\begin{minted}{custompython}
numero_secreto = 57
print("Tente adivinhar o número entre 0 e 99.")
palpite = int(input("Digite seu palpite: "))

while palpite != numero_secreto:
    if palpite < numero_secreto:
        print("O número secreto é maior. Tente novamente!")
    else:
        print("O número secreto é menor. Tente novamente!")

    palpite = int(input("Digite seu palpite: "))

print("Parabéns! Você ganhou!")
\end{minted}

Já no exemplo a seguir, o programa imprime no terminal os números naturais menores que \inlcode{10}.
(valor definido pelo usuário):
\begin{minted}{custompython}
contador = 0
while contador < 10:
    print(contador, end=' ')
    contador += 1
\end{minted}
\begin{minted}{text}
0 1 2 3 4 5 6 7 8 9
\end{minted}

Nesse contexto, a variável \inlcode{contador} desempenha o papel de variável de controle, sendo responsável
por acompanhar e atualizar o estado da repetição a cada iteração.

Como o loop \inlcode{while} apenas avalia uma expressão booleana, é necessário que essa variável seja inicializada, monitorada e
modificada manualmente.
Sua inicialização fora do laço e seu incremento em \inlcode{1} a cada iteração é essencial para garantir o
avanço do programa e evitar loops infinitos.

Esse processo de percorrer uma sequência pre-definida (ou até coleção de dados) é tão fundamental em programação
que a maioria das linguagens, incluindo Python, oferece uma estrutura específica para essa finalidade.
Com o comando \inlcode{for}, o estado da iteração é gerenciado automaticamente, sem
necessidade de controle manual pelo programador, tornando o código mais conciso e legível.

\subsection{\inlcode{for}}

O comando \inlcode{for} é utilizado principalmente para percorrer elementos de uma sequência,
garantindo que cada item seja processado de maneira previsível.

O exemplo anterior pode ser reescrito substituíndo o \inlcode{while} por um \inlcode{for} da seguinte forma:
\begin{minted}{custompython}
for i in range(10):
    print(i, end=' ')
\end{minted}
\begin{minted}{text}
0 1 2 3 4 5 6 7 8 9
\end{minted}

Aqui, a variável \inlcode{i} não precisa ser declarada ou incrementada manualmente, pois o próprio
comando \inlcode{for} atribui a ela cada valor dentro da sequência gerada por \inlcode{range(10)}.
Na primeira iteração, \inlcode{i} recebe o valor \inlcode{0}, na segunda \inlcode{1}, e assim por diante,
até atingir \inlcode{9}.
O valor \inlcode{10} é não inclusivo, ou seja, o loop termina antes de alcançá-lo, e já que \inlcode{10} gera uma
sequência de \inlcode{0} a \inlcode{9}, temos exatamente \inlcode{10} repetições do loop.

A estrutura \inlcode{for} é particularmente útil ao trabalhar com listas, tuplas, dicionários e outras coleções
de dados, permitindo percorrer diretamente seus elementos sem necessidade de manipular índices manualmente.
Além disso, o \inlcode{for} em Python é conceitualmente equivalente ao \inlcode{for-each} presente em linguagens
como Java e C\#, pois permite acessar diretamente cada elemento de um iterável, tornando o processo de iteração
mais intuitivo.

O exemplo a seguir demostra isso:
\begin{minted}{custompython}
nomes = ['Fulano', 'Beltrano', 'Sicrano']

for nome in nomes:
    print(f"Saudações, {nome}!")
\end{minted}
\begin{minted}{text}
Saudações, Fulano!
Saudações, Beltrano!
Saudações, Sicrano!
\end{minted}



Aqui, o loop percorre diretamente os elementos da lista \inlcode{nomes}, atribuindo cada valor individualmente
à variável \inlcode{nome} em cada iteração, sem exigir um índice explícito.
Dessa forma, o processamento ocorre um a um, garantindo que cada elemento seja tratado separadamente.
Essa abordagem torna o código mais claro, eliminando a necessidade de manipulação manual de índices e
facilitando a interação direta com os dados da coleção.

Em situações onde é necessário acessar tanto o índice quanto o valor de cada elemento durante a repetição, o método \inlcode{enumerate()} pode ser utilizado. Ele retorna pares de valores, onde cada elemento da coleção recebe seu respectivo índice de forma automática.
\begin{minted}{custompython}
nomes = ['Fulano', 'Beltrano', 'Sicrano']

for indice, nome in enumerate(nomes):
    print(f"{indice}: Saudações, {nome}!")
\end{minted}
\begin{minted}{text}
0: Saudações, Fulano!
1: Saudações, Beltrano!
2: Saudações, Sicrano!
\end{minted}

Python também oferece diversas funções embutidas para manipular iteradores, como \inlcode{reversed()},
permitem percorrer uma sequência de trás pra frente:
\begin{minted}{custompython}
nomes = ['Fulano', 'Beltrano', 'Sicrano']

for nome in reversed(nomes):
    print(f"Saudações, {nome}!")
\end{minted}
\begin{minted}{text}
Saudações, Sicrano!
Saudações, Beltrano!
Saudações, Fulano!
\end{minted}

Além disso, \inlcode{zip()} possibilita a iteração simultânea sobre múltiplas coleções, associando pares de valores
um a um a cada repetição:
\begin{minted}{custompython}
nomes = ['Fulano', 'Beltrano', 'Sicrano']
idades = [22, 35, 65]

for nome, idade in zip(nomes, idades):
    print(f"{nome} tem {idade} anos.")
\end{minted}
\begin{minted}{text}
Fulano tem 22 anos.
Beltrano tem 35 anos.
Sicrano tem 65 anos.
\end{minted}


Tanto o laço \inlcode{while} quanto o \inlcode{for} podem ser interrompidos a qualquer momento com o
comando \inlcode{break}.
Esse mecanismo permite finalizar a execução do loop imediatamente, ignorando qualquer
condição restante e continuando a execução do programa a partir da próxima instrução após o laço.










