\chapter*{Refs}

%\begin{enumerate}
%    \item Python Software Foundation. \textit{The Python Tutorial}.\\
%          Disponível em: \url{https://docs.python.org/3/tutorial/}
%
%    \item Python Software Foundation. \textit{Python Language Reference}.\\
%          Disponível em: \url{https://docs.python.org/3/reference/}
%
%    \item van Rossum, G.; Warsaw, B.; Coghlan, N. \textbf{PEP 8}: Style Guide for Python Code.\\
%          \url{https://peps.python.org/pep-0008/}
%
%    \item van Rossum, G.; Giannattasio, A. \textbf{PEP 563}: Postponed Evaluation of Annotations.\\
%          \url{https://peps.python.org/pep-0563/}
%
%    \item Brandl, B.; Fowler, J.; O’Connor, E. et al. \textbf{PEP 634–636}: Structural Pattern Matching.\\
%          \url{https://peps.python.org/pep-0634/}
%
%    \item Sweigart, A. \textit{Automate the Boring Stuff with Python} (2ª ed.). No Starch Press, 2019.
%
%    \item Matthes, E. \textit{Python Crash Course} (2ª ed.). No Starch Press, 2019.
%
%    \item Lutz, M. \textit{Learning Python} (5ª ed.). O’Reilly Media, 2013.
%
%    \item Real Python. \url{https://realpython.com}
%
%    \item Programiz – \textit{Python Programming Tutorial}. \url{https://www.programiz.com/python-programming}
%
%    \item W3Schools – \textit{Python Tutorial}. \url{https://www.w3schools.com/python/}
%
%    \item Coutinho, E. \textit{Referência Rápida de Python} (Blog, pt-BR).\\
%          \url{https://educoutinho.com.br/python-referencia-rapida}
%
%    \item Pyright – Static Type Checker for Python (VS Code Docs).\\
%          \url{https://github.com/microsoft/pyright}
%
%    \item JetBrains. \textit{PyCharm Guide}. \url{https://www.jetbrains.com/pycharm/guide}
%
%    \item Microsoft. \textit{Python in Visual Studio Code}.\\
%          \url{https://code.visualstudio.com/docs/languages/python}
%
%    \item cheat-sheets.org – \textit{Python Cheat Sheet}.\\
%          \url{https://cheat-sheets.org/#Python}
%
%    \item MIT OpenCourseWare. 6.0001 – Introduction to Computer Science and Programming in Python.\\
%          \url{https://ocw.mit.edu/courses/6-0001-introduction-to-computer-science-and-programming-in-python}
%
%    \item Harvard University. CS50P – \textit{Introduction to Programming with Python}.\\
%          \url{https://cs50.harvard.edu/python}
%\end{enumerate}








\begin{table}[htbp]
\centering
\renewcommand{\arraystretch}{1.2}
\begin{tabular}{>{\raggedright}p{3.0cm} >{\raggedright}p{5.5cm} >{\raggedright\arraybackslash}p{6.5cm}}
\toprule
\textbf{Categoria} & \textbf{Exemplos de obras / links} & \textbf{Por que são candidatos fortes?} \\ \midrule
Documentação oficial &
\begin{itemize}
  \item \url{https://docs.python.org/3/tutorial/} (The Python Tutorial)
  \item \url{https://docs.python.org/3/reference/} (Language Reference)
\end{itemize} &
Sumário do PDF espelha a sequência “tipos → coleções → funções → controle de fluxo” usada no tutorial oficial, além de empregar terminologia canônica. \\ \midrule

Livros introdutórios &
\begin{itemize}
  \item “Automate the Boring Stuff with Python” – A. Sweigart
  \item “Python Crash Course” – E. Matthes
  \item “Learning Python” – M. Lutz
\end{itemize} &
Exemplos de variáveis simples, listas e tuplas são muito semelhantes aos encontrados nesses best-sellers para iniciantes. \\ \midrule

Tutoriais on-line gratuitos &
\begin{itemize}
  \item Real Python (\url{https://realpython.com})
  \item Programiz (\url{https://programiz.com/python-programming})
  \item W3Schools Python
\end{itemize} &
Frases como “coleção mutável e ordenada” e “set não mantém ordem fixa” aparecem quase literais nesses sites. \\ \midrule

Blogs em português &
“Referência rápida de Python” – Edu Coutinho Blog &
A seção sobre IDEs e PATH repete recomendações comuns em blogs PT-BR voltados a iniciantes. \\ \midrule

Documentação de IDEs &
\begin{itemize}
  \item PyCharm Guide (JetBrains)
  \item “Python in VS Code” docs
\end{itemize} &
O PDF descreve autocomplete, refatoração e integração com Git quase palavra-por-palavra dos materiais oficiais das IDEs. \\ \midrule

Cursos acadêmicos abertos &
\begin{itemize}
  \item MIT OCW 6.0001
  \item Harvard CS50P
\end{itemize} &
A ordem curricular “variáveis → coleções → funções → controle de fluxo” segue o esqueleto desses cursos, frequentemente copiado em apostilas universitárias. \\ \bottomrule
\end{tabular}

\end{table}
