\chapter{Título do capítulo}
\lipsum[1-2]

\section{Título da seção}
\lipsum[3-5]
%

\section{Equações}
Exemplo de equação:
    $$
    f(t) = \sum_{h=1,3\ldots}^\infty \dfrac{1}{h!} \sin(h \omega t)
    $$

Outro exemplo:
\begin{equation}
    \begin{bmatrix}
        d \\
        q \\
        0
    \end{bmatrix}
    =
    \frac{2}{3}
    \begin{bmatrix}
        \cos\theta & \cos(\theta - 120^\circ) & \cos(\theta + 120^\circ) \\
        \sin\theta & \sin(\theta - 120^\circ) & \sin(\theta + 120^\circ) \\
        1/2 & 1/2 & 1/2
    \end{bmatrix}
    \begin{bmatrix}
        a \\
        b \\
        c
    \end{bmatrix}
    \label{eq:abs_dq0}
\end{equation}

\begin{equation}
    \begin{bmatrix}
        d \\
        q
    \end{bmatrix}
    =
    \begin{bmatrix}
        \cos\theta & \sin\theta \\
        -\sin\theta & \cos\theta
    \end{bmatrix}
    \begin{bmatrix}
        \alpha \\
        \beta
    \end{bmatrix}
    \label{eq:parke}
\end{equation}




\section{Figuras}
\lipsum[6]

Exemplo de gráfico utilizando \texttt{tikz}:
\begin{figure}[h]
    \centering
    \begin{tikzpicture}
    \begin{axis}[
        width=280px,
        height=160px,
        axis lines = middle,
        enlargelimits = true,
        xlabel = $x$,
        ylabel = $y$,
        legend pos = north east,
        grid = major,
        legend style={fill=white, draw=gray, text=black},
        yticklabel style={xshift=-20pt},
        axis line style={thin},
        grid style={solid, gray!90, line width=0.2pt}
    ]
        \addplot [primary, thick, domain=0:10, samples=100]
            {exp(-0.4*x) * sin(deg(x))};
        \addlegendentry{$e^{-0.4x} \sin(x)$}

        \addplot [secondary, thick, dashed, domain=0:10, samples=100]
            {exp(-0.2*x) * cos(deg(x))};
        \addlegendentry{$e^{-0.2x} \cos(x)$}
    \end{axis}
\end{tikzpicture}
    \caption{Exemplo de figura \texttt{tikz}}
    \label{fig:exemplo}
\end{figure}


\lipsum[7]




Exemplo de imagem a partir de arquivo apresentado na \figr{fig:exemplo1}.
\begin{figure}[h]
    \centering
    \includegraphics[width=0.3\textwidth, angle=20]{../class/figs/logo_capa}
    \caption{Exemplo de importação de figura}
    \label{fig:exemplo1}
\end{figure}

Exemplo de subfigures apresentado na \ref{fig:figuras_conjunto}, que possui as sub-figuras \ref{fig:imagem1} e \ref{fig:imagem2}.
\begin{figure}[h]
    \centering
    \begin{subfigure}{0.35\textwidth}
        \centering
        \includegraphics[width=\linewidth, angle=90]{../class/figs/logo_capa}
        \caption{Legenda da primeira figura}
        \label{fig:imagem1}
    \end{subfigure}
    \hspace{15mm}
    \begin{subfigure}{0.35\textwidth}
        \centering
        \includegraphics[scale=0.45, angle=45]{../class/figs/logo_capa}
        \caption{Legenda da segunda figura}
        \label{fig:imagem2}
    \end{subfigure}
    \caption{Legenda geral do conjunto de figuras}
    \label{fig:figuras_conjunto}
\end{figure}



Exemplo de imagem a partir de arquivo com anotação \texttt{tikz} apresentado na \figr{fig:exemplo2}.
\begin{figure}[h]
    \centering
    \begin{tikzpicture}
        \node[anchor=south west] (img) at (0,0) {\includegraphics[scale=0.5]{../class/figs/logo_capa}};
        \draw[secondary, thick, line cap=round, <-] (3,1.2) -- (3.5,1.4) node[right, draw=secondary, rounded corners, thin, fill=black!85!secondary] {nota importante!};
    \end{tikzpicture}
    \caption{Exemplo de figura com anoração co. \texttt{tikz}}
    \label{fig:exemplo2}
\end{figure}



\section{Código fonte}
\lipsum[8]

Exemplo de código fonte usando ambiente \texttt{minted}:
\begin{minted}{custompython}
'''Exemplo de código definido no proprio .tex com \begin{minted}{custompython}...'''
def exemplo():
    pass
\end{minted}

\lipsum[9-11]

Exemplo de código fonte importado de arquivo usando \texttt{inputminted}:
\inputminted{custompython}{./capitulos/ex_codigo.py}
%\inputminted[linenos, firstnumber=5478]{custompython}{./capitulos/ex_codigo.py}


\section{Citações}
A referência \cite{einstein1905} é um exemplo de citação.
