\chapter{Classes}\label{class}

Além dos tipos primitivos e coleções introduzidos no Capítulo \ref{vars}, podemos ainda criar nossos próprios tipos personalizados.

Uma classe descreve um tipo composto que reúne, em um único objeto, um conjunto coerente de variáveis (atributos) e as
funções que operam sobre elas (métodos), para a manipulação desses dados.


Embora as classes sejam um dos pilares da Programação Orientada a Objetos (OOP), viabilizando conceitos como encapsulamento,
abstração, herança e polimorfismo, este texto introdutório não se propõe a discutir, muito menos advogar por OOP como
abordagem universal.

Ainda assim, o uso de classes pode melhorar significativamente a organização do código, mesmo sem aderir estritamente
aos princípios da OOP.

Além disso, compreender a estrutura básica de classes em Python é essencial, especialmente para entender e utilizar
pacotes especializados, que serão explorados em capítulos futuros.

Nada melhor do que demonstrar na prática as vantagens do uso de classes.

Para isso, vamos revisitar um dos exemplos apresentados no Capítulo \ref{iffor}, onde criamos uma função para buscar
o contato de um usuário a partir de seu cadastro.

O código original é replicado abaixo:

\begin{minted}{custompython}
def obter_contato(cadastro: dict) -> str:
    if 'email' in cadastro and cadastro['email'] != '':
        return cadastro['email']
    elif 'telefone' in cadastro and cadastro['telefone'] != '':
        return cadastro['telefone']
    else:
        return 'nenhum contato cadastrado'

usuario1 = {'nome': 'Fulano', 'email': 'fulano@ifsc.edu.br', 'telefone': '(48)99999-9999'}
usuario2 = {'nome': 'Beltrano', 'telefone': '(48)99999-8888'}
usuario3 = {'nome': 'Sicrano', 'telefone': ''}

print('contato usuario1:', obter_contato(usuario1))
print('contato usuario2:', obter_contato(usuario2))
print('contato usuario3:', obter_contato(usuario3))
\end{minted}
\begin{minted}{text}
contato usuario1: fulano@ifsc.edu.br
contato usuario2: (48)99999-8888
contato usuario3: nenhum contato cadastrado
\end{minted}

Note que a função \inlcode{obter_contato()} possui um único parâmetro \inlcode{cadastro}, que é um dicionário com
algumas chaves \inlstr{nome}, \inlstr{telefone} e \inlstr{email}.
Nesse caso, o uso de um dicionário foi a opção encontrada para agrupar os dados relacionados a um indivíduo,
fornecendo uma estrutura básica para armazenar essas informações sem que fiquem dispersas em múltiplas variáveis.

Como os valores associados às chaves \inlstr{nome}, \inlstr{telefone} e \inlstr{email} estão logicamente interligados,
faz sentido agrupá-los.
No entanto, há limitações nessa abordagem pois a função \inlcode{obter_contato()} só funciona corretamente se o
dicionário contiver as chaves esperadas.
O código não protege contra inconsistências ou dados ausentes de forma organizada.
Fica a cardo chamador da função manter uma organização e consistência na estrutura do dicionário fornecido.

Uma forma mais estruturada de lidar com essas questões é utilizar classes, que encapsulam dados
(\inlstr{nome}, \inlstr{telefone}, \inlstr{email}) e funcionalidade (\inlcode{obter_contato()}) em um único objeto.


Poderíamos refatorar o programa criando um novo tipo personalizado chamado \inlcode{Usuario}:
\begin{minted}[linenos]{custompython}
# definição da classe
class Usuario:
    # o método __init__ é executado ao criar um novo objeto
    def __init__(self, nome: str, telefone: str = '', email: str = ''):
        self.nome = nome
        self.telefone = telefone
        self.email = email

    def contato(self) -> str:
        if self.email:
            return self.email
        elif self.telefone:
            return self.telefone
        else:
            return 'nenhum contato cadastrado'

# criando instâncias da classe Usuario
usuario1 = Usuario(nome='Fulano', email='fulano@ifsc.edu.br', telefone='(48)99999-9999')
usuario2 = Usuario(nome='Beltrano', telefone='(48)99999-8888')
usuario3 = Usuario(nome='Sicrano')

# utilizando o método contato() e apresentando os resultados
print('contato usuario1:', usuario1.contato())
print('contato usuario2:', usuario2.contato())
print('contato usuario3:', usuario3.contato())
\end{minted}
\begin{minted}{text}
contato usuario1: fulano@ifsc.edu.br
contato usuario2: (48)99999-8888
contato usuario3: nenhum contato cadastrado
\end{minted}

Vamos explorar as partes fundamentais desse código linha por linha.

Na \textbf{linha 2} temos a declaração que estamos definindo uma nova \inlcode{class}
chamada \inlcode{Usuario}\footnote{Diferente da convenção utilizada para funções e variáveis, que geralmente seguem o
estilo \textit{snake\_case} (\inlcode{exemplo_variavel}), nomes de classes costumam ser escritos em \textit{CamelCase} (\inlcode{NomeDaClasse}).}.

Essa definição consiste em dois métodos, ou duas funções associadas à classe: \magic{__init__()} (\textbf{linha 4}) e
\inlcode{contato()} (\textbf{linha 9}).

O método \magic{__init__()} é o construtor da classe, que é executado automaticamente ao criarmos um novo objeto do
tipo \inlcode{Usuario} (o que é feito na \textbf{linha 18}).
Esse processo é conhecido como instanciação da classe, e resulta na criação de um objeto em memória.
Enquanto a classe é apenas uma definição da estrutura e do comportamento dos objetos (um \emph{blueprint}), a instância
representa a consolidação de um objeto desse tipo em memória, com valores próprios.


Assim como funções, os métodos podem ter parâmetros que podem ser obrigatórios ou opcionais (recebendo
valores padrão).
No caso de \magic{__init__()}, os parâmetros são \magic{self}, \inlcode{nome}, \inlcode{email} e \inlcode{telefone}.

A principal diferença desse método quando comparado a uma função padrão, é a presença do parâmetro
\magic{self}, que é uma referência à própria instância do objeto\footnote{O nome \magic{self} é apenas uma convenção, e sua função de autoreferência à instância do objeto está
diretamente relacionada à posição do parâmetro, que deve sempre ser o primeiro em qualquer método.
Independentemente do nome utilizado, esse primeiro parâmetro representa o próprio objeto, permitindo o acesso e a
manipulação de seus atributos e métodos. Contudo, usar \magic{self} é altamente recomendado, pois é uma prática amplamente
reconhecida e melhora a legibilidade do código.}.
Não precisamos fornecer \magic{self} a um método de forma manual, o
Python faz isso de forma automática e nos fornece \magic{self} justamente para podermos manipular os dados relativos à instância.

Na \textbf{linha 5}, criamos o atributo \magic{self.}\inlcode{nome}, atribuindo a ele o valor do parâmetro \inlcode{nome}.

Um atributo é uma variável interna da instância da classe, responsável por armazenar informações específicas de cada
objeto criado.
São esses atributos que compõem o objeto.

Embora o atributo e o parâmetro tenham o mesmo nome (o que é uma prática comum, mas não necessária), eles são variáveis
distintas.
O parâmetro \inlcode{nome} é passado ao método \magic{__init__()} no momento da criação de um novo objeto da classe
\inlcode{Usuario}, e seu valor é armazenado internamente no atributo \magic{self.}\inlcode{nome}, garantindo que essa
informação fique acessível ao longo da vida útil do objeto.

Além de \inlcode{nome}, são criados mais dois atributos para armazenar \inlcode{email} e \inlcode{telefone}.
Esses três atributos formam a estrutura de dados armazenados em memória para um objeto da classe \inlcode{Usuario}.


Pulando para a \textbf{linha 18}, que na realidade é a primeira a ser executada (já que as anteriores pertencem apenas
à definição da classe \inlcode{Usuario}), temos a criação da variável \inlcode{usuario1}.

À direita do operador de atribuição (\inlcode{=}) há uma chamada à classe \inlcode{Usuario()}, o que significa que um
novo objeto do tipo \inlcode{Usuario} está sendo instanciado.

Ao chamar \inlcode{Usuario(nome='Fulano', email=...)}, os argumentos
fornecidos (\inlcode{nome}, \inlcode{email} e \inlcode{telefone}) são passados para o método \magic{__init__()}, que
então atribui esses valores aos atributos internos do objeto.
Isso resulta na criação de uma instância única, armazenada na variável \inlcode{usuario1}, contendo todas as
informações fornecidas.

Nas \textbf{linhas 19} e \textbf{20}, criamos mais duas instâncias da classe \inlcode{Usuario}, com valores distintos.
Cada instância da classe representa um objeto único na memoria, contendo suas próprias informações
de \inlcode{nome}, \inlcode{telefone} e \inlcode{email}, mas possuem a mesma estrutura definida pela classe.

Na \textbf{linha 23}, dentro do comando \inlcode{print()}, temos a chamada  \inlcode{usuario1.contato()}, que executa o
método \inlcode{contato()} definido na \textbf{linha 9}.

Um detalhe importante é que, na definição de \inlcode{contato(}\magic{self}\inlcode{)}, declaramos um parâmetro
\magic{self}, mas não o fornecemos explicitamente na chamada do método.

Isso ocorre porque o Python passa \magic{self} automaticamente, garantindo que o método seja chamado sobre a instância correta.
Ou seja, a chamada \inlcode{usuario1.contato()} equivale internamente a \inlcode{contato(usuario1)}, onde \magic{self} recebe
\inlcode{usuario1} como argumento.

Note que o encapsulamento dos dados e funcionalidades contribui para a organização e legibilidade do código.
As instâncias de classes geralmente são objetos mutáveis, como os abordados no Capítulo \ref{vars}.
Na realidade, em Python, todos os tipos de dados são implementados como classes, até mesmo os tipos primitivos como inteiros.

Diferente de outras linguagens como Java, C++, C\# e Rust, onde existe um sistema explícito de controle de
acesso (como \inlcode{private}, \inlcode{protected} e \inlcode{public}), em Python todos os membros
de uma classe são públicos por padrão.

Isso significa que qualquer parte do código que tenha acesso a uma instância da classe pode ler e modificar diretamente
seus atributos, além de chamar qualquer método definido na classe.

Embora Python não imponha um mecanismo formal de privacidade, existe uma convenção para indicar atributos
internos\footnote{Convencionou-se utilizar um underscore (\inlcode{_}) antes do nome do atributo para indicar que ele
não deve ser acessado diretamente (ex: \inlcode{self._informacao_privada}). Apesar de ainda ser possível modificar
esses atributos externamente, essa prática sinaliza que o atributo é de uso interno, recomendando que sua manipulação
ocorra apenas dentro da própria classe por seus métodos. IDEs e ferramentas de desenvolvimento costumam alertar sobre
tentativas de acesso direto a esses atributos, reforçando essa boa prática.}.



O exemplo a seguir demostra esse acesso a atributos:
\begin{minted}{custompython}
class Usuario:
    def __init__(self, nome: str, telefone: str = '', email: str = ''):
        self.nome = nome
        self.telefone = telefone
        self.email = email

usuario = Usuario(nome='Fulano', email='fulano@ifsc.edu.br', telefone='(48)99999-9999')
print(f"{usuario.nome}, {usuario.email}, {usuario.telefone}")   # lendo atributos

usuario.email = 'fulano_de_tal@ifsc.edu.br'   # modificando atributo email
usuario.telefone = '(11)88888-9999'           # modificando atributo telefone
print(f"{usuario.nome}, {usuario.email}, {usuario.telefone}")    # lendo atributos
\end{minted}
\begin{minted}{text}
Fulano, fulano@ifsc.edu.br, (48)99999-9999
Fulano, fulano_de_tal@ifsc.edu.br, (11)88888-9999
\end{minted}


As classes em Python oferecem diversas funcionalidades avançadas, como herança, sobrecarga de operadores, métodos
mágicos, polimorfismo, abstração, propriedades e metaclasses, que ampliam o suporte à programação orientada a objetos.

Embora sejam conceitos importantes, este tutorial introdutório foca nos princípios básicos.
O aprofundamento nesses temas fica a critério do leitor, que poderá explorá-los de forma gradativa, em um momento
oportuno, conforme sua necessidade e evolução no aprendizado.

Para praticar os conceitos de classes, refatore o código da questão \inlcode{tictactoe}, transformando o parâmetro
\inlcode{board_state} em um atributo e a função \inlcode{eval\_move()} método
da classe \inlcode{TicTacTtoe}.



