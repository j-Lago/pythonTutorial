Escreva uma função \inlcode{balanceamento(valores: list) -> tuple[list, list]} que receba uma lista numérica de
tamanho arbitrário, calcule a média de seus valores e, com base nisso, retorne duas novas listas:
\begin{itemize}
    \item a primeira contendo os elementos da lista original cujos valores são menores ou iguais à média;
    \item a segunda contendo os elementos estritamente maiores que a média.
\end{itemize}

Note que, em casos particulares, uma ou ambas as listas resultantes podem estar vazias.

Implemente
\begin{minted}{custompython}
def balanceamento(valores: list) -> tuple[list, list]:
    # seu código aqui

l, g = balanceamento([2, 4, 8, 3, 0, 3, 5, 7])
print(f"{l = }")
print(f"{g = }")
\end{minted}

Resultado esperado:
\begin{minted}{text}
l = [2, 4, 3, 0, 3]
g = [8, 5, 7]
\end{minted}

\subsection*{Desafio:}
Modifique a função para que ela aceite, opcionalmente, um segundo parâmetro com pesos:
\inlcode{balanceamento(valores: list, pesos: list | None = None) -> tuple[list, list]}

Neste caso, \inlcode{pesos} deve ser uma lista de mesmo tamanho que \inlcode{valores}, representando o peso associado
a cada elemento.

O objetivo agora é calcular a média ponderada dos valores, considerando os pesos fornecidos, e utilizá-la como
critério para particionar os elementos em duas listas:
\begin{itemize}
    \item a primeira lista deve conter os elementos cujo produto \inlcode{valor * peso} seja menor ou igual à
    média ponderada;
    \item a segunda lista deve conter os elementos cujo produto \inlcode{valor * peso} seja estritamente maior que a
    média ponderada.
\end{itemize}

Se o parâmetro \inlcode{pesos} não for fornecido (\inlcode{None}), a função deve atribuir peso \inlcode{1} a todos
os elementos, recuperando o comportamento original com média aritmética simples.

A função deve lançar uma exceção caso \inlcode{valores} e \inlcode{pesos} não tenham o mesmo tamanho ou se algum dos
argumentos não seja uma lista numérica.

Implemente:
\begin{minted}{custompython}
def balanceamento(valores: list, pesos: list | None = None) -> tuple[list, list]:
    # seu código aqui

valores = [8, 3, 11, 6, 4, 9]
pesos  =  [1.0, 0.8, 1.2, 0.9, 1.1, 1.0]

l, g = balanceamento(valores, pesos)

print(f"{l = }")
print(f"{g = }")
\end{minted}

Resultado esperado:
\begin{minted}{text}
l = [3, 6, 4]
g = [8, 11, 9]
\end{minted}









