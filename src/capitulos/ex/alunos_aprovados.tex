Crie uma função que receba um dicionário de alunos e suas notas e retorne uma lista contendo apenas os nomes dos alunos
aprovado (nota mínima 6.0).
\begin{minted}{custompython}
def alunos_aprovados(dados):
    # seu código aqui

alunos = {
    'Alice': 8.5,
    'Bruno': 5.2,
    'Carlos': 9.0,
    'Maria': 7.1,
    'Pedro': 5.8
}
aprovados = alunos_aprovados(alunos)
print(f"Aprovados: {aprovados}")
\end{minted}

Resultado esperado:
\begin{minted}{text}
Aprovados: ['Alice', 'Carlos', 'Maria']
\end{minted}

\subsection*{Desafio:}
Substitua a linha contendo o \inlcode{print} para uma chamada a uma nova função (a ser desenvolvida por você) que
recebe a lista de aprovados e não retorna nada, mas que imprima os nomes dos aprovados de forma mais legível, com apenas
um nome um por linha.
Resultado esperado:
\begin{minted}{text}
Aprovados:
  Alice
  Carlos
  Maria
\end{minted}




