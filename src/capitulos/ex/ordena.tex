Escreva uma função  \inlcode{ordena(valores)}, que recebe uma lista de números inteiros de tamanho arbitrário e ordena
seus elementos em ordem crescente.
A função deve retornar duas listas: a primeira contendo os valores ordenados da lista original; a segunda contendo os
índices dos elementos da lista original que correspondem à posição dos valores ordenados.

\begin{minted}{custompython}
def ordena(valores: list[int]) -> tuple[list[int], list[int]]:
    # seu código aqui

vord, ids = ordena([1, 4, 3, 5, 8, 2])
print(f"vord: {vord}\n ids: {ids}")
\end{minted}

Resultado esperado:
\begin{minted}{text}
vord: [1, 4, 3, 5, 8, 2]
 ids: [4, 3, 1, 2, 5, 0]
\end{minted}

Não utilize a função \inlcode{sort()} ou \inlcode{sorted()} do Python.
Implemente sua própria lógica de ordenação.

Depois de implementar seu própria algoritmo, utilize as soluções embutidas da linguagem.

