Escreva uma função  \inlcode{procura(lista, valor)},  que recebe uma lista de números de tamanho arbitrário e um valor
a ser buscado dentro dessa lista.
A função deve retornar uma nova lista contendo os índices das ocorrências do valor na lista original.

Se o valor não for encontrado, a função deve retornar uma lista vazia.
\begin{minted}{custompython}
def procura(lista, valor):
    # seu código aqui

ids = procura([2, 4, 8, 3, 0, 3, 5, 7], 3)
print(ids)
\end{minted}

Resultado esperado:
\begin{minted}{text}
[3, 5]
\end{minted}

\subsection*{Desafio:}
Além de criar sua própria lógica para a busca, tente também resolver o problema utilizando o função embutida \inlcode{filter()}.
Para isso, você precisará aplicar uma função \inlcode{lambda}, que não foi abordada neste tutorial.
Pesquise sobre essa técnica e descubra como utilizá-la para filtrar os índices corretamente!


