\newcommand{\po}{{\color{MyDarkGreen}$O$}}
\newcommand{\px}{{\color{MyBrickRed}\ding{55}}}
\definecolor{MyDarkGreen}{rgb}{0.0,0.4,0.0}
\definecolor{MyBrickRed}{rgb}{0.8, 0.25, 0.33}

Escreva uma função \inlcode{eval_move(board_state, next_move)} que processe a lógica do jogo tic-tac-toe (jogo da velha)
e informa o resultado da partida.

   O primeiro argumento recebido pela função (\inlcode{board\_state}) é uma matriz $3\times3$ (implementada na forma de
\inlcode{list[list[int]]}) que contém o estado atual do tabuleiro.
Elementos da matriz contendo \inlcode{0} indicam que aquela posição ainda não foi jogada.
Posições da matriz contendo \inlcode{1} indicam que ela foi tomada pelo jogador \px, enquanto posições
contendo \inlcode{-1} foram tomadas pelo jogador \po.

Valores diferentes de \inlcode{-1}, \inlcode{0} e \inlcode{1} constituem um estado inválido do tabuleiro.
Outras formas de estados inválidos também devem ser considerados.
Por exemplo, muitas posições tomadas por um mesmo jogador ou o jogador \po~com mais posições tomadas que \px~(considerando que \px~sempre inicia jogando).
Abaixo são apresentados alguns valores de \inlcode{board\_state} e o que cada um deles representa.

\begin{minipage}{0.25\textwidth}
\begin{minted}{custompython}
# valid state
board_state = [
    [ 1, -1,  0],
    [ 0,  1,  1],
    [-1,  0, -1]
]
\end{minted}
\end{minipage}
%
\begin{minipage}{0.2\textwidth}
{\renewcommand{\arraystretch}{1.45}
\begin{center}
\begin{tabular}{ c | c | c }
 \px & \po &  \\ \hline
  & \px &  \px\\ \hline
 \po &  & \po
\end{tabular}
\end{center}
}
\end{minipage}
%
%
%
\hspace{0.07\textwidth}
\begin{minipage}{0.25\textwidth}
\begin{minted}{custompython}
# invalid state
board_state = [
    [ 1, -1,  0],
    [ 0, -1,  1],
    [-1,  0, -1]
]
\end{minted}
\end{minipage}
%
\begin{minipage}{0.2\textwidth}
{\renewcommand{\arraystretch}{1.45}
\begin{center}
\begin{tabular}{ c | c | c }
 \px & \po &  \\ \hline
  & \po &  \px\\ \hline
 \po &  & \po
\end{tabular}
\end{center}
}
\end{minipage}


\begin{minipage}{0.25\textwidth}
\begin{minted}{custompython}
# player x wins!
board_state = [
    [ 1, -1, -1],
    [ 0,  1,  1],
    [-1,  0,  1]
]
\end{minted}
\end{minipage}
%
\begin{minipage}{0.2\textwidth}
{\renewcommand{\arraystretch}{1.45}
\begin{center}
\begin{tabular}{ c | c | c }
 \px & \po &  \po\\ \hline
  & \px &  \px\\ \hline
 \po &  & \px
\end{tabular}
\end{center}
}
\end{minipage}
%
%
%
\hspace{0.07\textwidth}
\begin{minipage}{0.25\textwidth}
\begin{minted}{custompython}
# draw!
board_state = [
    [ 1, -1,  1],
    [-1, -1,  1],
    [ 1,  1, -1]
]
\end{minted}
\end{minipage}
%
\begin{minipage}{0.2\textwidth}
{\renewcommand{\arraystretch}{1.45}
\begin{center}
\begin{tabular}{ c | c | c }
 \px & \po & \px \\ \hline
  \po& \po &  \px\\ \hline
 \px &  \px & \po
\end{tabular}
\end{center}
}
\end{minipage}


O segundo argumento recebido (\inlcode{next\_move}) é uma tupla de dois inteiros \inlcode{(l,c)}, indicando
respectivamente a linha e a coluna jogada pelo próximo jogador.
Valores de \inlcode{l} e \inlcode{c} diferentes de \inlcode{0}, \inlcode{1} e \inlcode{2} constituem uma tentativa de joga inválida.
Também é uma jogada inválida qualquer posição \inlcode{(l,c)} previamente ocupada.

A função deve inferir qual dos jogadores (\px~ou~\po) está realizando a jogada \inlcode{next\_move} a partir do
estado do tabuleiro (\inlcode{board\_state}), garantindo a alternância entre os jogadores.
O jogador \px~joga primeiro.

   Uma vez validado o estado atual, o jogador da vez e a posição a ser jogada por este, a função \inlcode{eval\_move()}
deve modificar o estado do tabuleiro executando a jogada solicitada (\emph{side-effect function}).
Se a jogada solicitada ou a posição prévia do tabuleiro for inválida, a função mantém o \texttt{board\_state} inalterado.

Alem de modificar \inlcode{board\_state}, a função também retorna um valor  (\inlcode{outcome}), indicando se a
jogada ou posição são inválidas, se houve um vencedor, se houve empate ou se o jogo ainda não terminou.
O retorno \inlcode{outcome} deve ser um inteiro que codifica essas situações da seguinte forma:
\begin{itemize}
\item \inlcode{-4} se a posição do tabuleiro passado como argumento for ilegal;
\item \inlcode{-2} se a próxima jogada for ilegal;
\item \inlcode{ 0} se a jogada foi executada com sucesso, mas ainda não houve um ganhador;
\item \inlcode{ 1} se a jogada foi executada com sucesso e o jogador \px~venceu;
\item \inlcode{ 2} se a jogada foi executada com sucesso e o jogador \po~venceu;
\item \inlcode{ 3} se a jogada foi executada com sucesso, mas o jogou terminou em empate.
\end{itemize}

Complete:
\begin{minted}{custompython}
def eval_move(board_state: list[list[int]], next_move: tuple[int, int]) -> int:
    #seu código aqui

board_state = [
    [ 1, -1,  0],
    [ 0,  0,  1],
    [-1,  0,  0]
]
outcome = eval_move(board_state, (1, 1))
print(f"{board_state = }, {outcome = }")

outcome = eval_move(board_state, (2, 2))
print(f"{board_state = }, {outcome = }")

outcome = eval_move(board_state, (0, 1))
print(f"{board_state = }, {outcome = }")

outcome = eval_move(board_state, (1, 0))
print(f"{board_state = }, {outcome = }")
\end{minted}

Resultado esperado:
\begin{minted}{text}
board_state = [[1, -1, 0], [0, 1, 1], [-1, 0, 0]], outcome = 0
board_state = [[1, -1, 0], [0, 1, 1], [-1, 0, -1]], outcome = 0
board_state = [[1, -1, 0], [0, 1, 1], [-1, 0, -1]], outcome = -2
board_state = [[1, -1, 0], [1, 1, 1], [-1, 0, -1]], outcome = 1
\end{minted}

Note que a função deve realizar várias ações:
\begin{itemize}
\item Validação do estado atual do tabuleiro;
\item Determinação do próximo jogador;
\item Verificação da próxima jogada;
\item Aplicação da jogada no tabuleiro;
\item Avaliação do resultado (vitória, empate ou jogo em andamento).
\end{itemize}

Convém dividir a implementação da função \inlcode{eval_move} em funções menores, dedicadas a cada uma dessas
sub-tarefas (nomeie essas funções com nomes significativos).

