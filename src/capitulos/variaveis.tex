\chapter{Variáveis e Tipos}

A computação em seu nível mais básico baseia-se na manipulação de dados armazenados na memória.

As linguagens naturais são ambíguas e, muitas vezes, subjetivas, o que pode gerar interpretações variadas.
Diferentemente delas, uma linguagem de programação deve ser precisa e determinística, garantindo previsibilidade no
processamento da informação.

Através dessas linguagens, comunicamos ao hardware como manipular os dados armazenados na memória, transformando-os
em novos valores.
Esses valores, por sua vez, são interpretados e associados a algo significativo no mundo real, seja um número, um
texto ou uma representação visual.

Blocos de dados armazenados na memória que representam coletivamente um conceito comum são chamados objetos.
Os tipos associados a esses objetos formam uma camada de abstração que define como um conjunto de bits deve ser
interpretado.
Essa abstração permite representar diversos conceitos, como números inteiros, valores fracionários, caracteres e outros
tipos de dados, tornando a manipulação dessas informações mais estruturada e intuitiva.

Diferente de algumas linguagens, em Python não é necessário declarar o tipo da variável explicitamente
--- o interpretador determina automaticamente o tipo com base no valor atribuído.

Python possui diversos tipos primitivos, que representam os dados mais básicos da linguagem.
Estes podem ser utilizados diretamente para expressar algum conceito real ou combinados para formar tipos compostos
(ou estruturados), capazes de representar ideias e relações mais complexas.


Os principais são:

\begin{itemize}
    \item \inlcode{int}: utilizado para armazenar números inteiros, como \inlcode{10}, \inlcode{-5}, e \inlcode{0}
    \item \inlcode{float}: representa valores com casas decimais, como \inlcode{3.14}, \inlcode{-0.5}, \inlcode{1.6e-3}
    \item \inlcode{complex}: representa números complexos, como \inlcode{3.2 + 4.7j}
    \item \inlcode{bool}: representa uma grandeza booleana, que pode assumir apenas \inlcode{True} ou \inlcode{False}, essencial para estruturas de decisão
    \item \inlcode{str}: string utilizado para armazenar textos, como \inlstr{Olá, mundo!}.
    \item \inlcode{NoneType}: pode assumir apenas o valor \inlcode{None}, e pode ser explicitamente atribuído
    a uma variável para sinalizar a ausência de um valor.



\end{itemize}

Em código a criação de variáveis com os tipos acima descritos ficaria:
\begin{minted}{custompython}
idade = 25
peso = 78.6
impedancia = 5 + 3j
ativo = True
nome = 'Fulano de Tal'
sem_valor = None

print(f"'idade' é uma variável do tipo {type(idade)} e valor: {idade}")
print(f"'peso' é uma variável do tipo {type(peso)} e valor: {peso}")
print(f"'impedancia' é uma variável do tipo {type(impedancia)} e valor: {impedancia}")
print(f"'ativo' é uma variável do tipo {type(ativo)} e valor: {ativo}")
print(f"'nome' é uma variável do tipo {type(nome)} e valor: {nome}")
print(f"'sem_valor' é uma variável do tipo {type(sem_valor)} e valor: {sem_valor}")
\end{minted}

O código acima apenas cria diversas variáveis de diferentes tipos e para cada uma delas imprime o seu tipo e valor no terminal:
\begin{minted}{text}
'idade' é uma variável do tipo <class 'int'> e valor: 25
'peso' é uma variável do tipo <class 'float'> e valor: 78.6
'impedancia' é uma variável do tipo <class 'complex'> e valor: (5+3j)
'ativo' é uma variável do tipo <class 'bool'> e valor: True
'nome' é uma variável do tipo <class 'str'> e valor: Fulano de Tal
'sem_valor' é uma variável do tipo <class 'NoneType'> e valor: None
\end{minted}

Antes de prosseguirmos e começarmos a utilizar essas variáveis para realizar tarefas úteis, é importante refletir
sobre o conceito de tipagem e as diferentes formas como as linguagens de programação a tratam.

Algumas linguagens possuem tipagem dinâmica, onde o tipo das variáveis é determinado durante a execução do programa.
Outras adotam tipagem estática, exigindo que os tipos sejam declarados antecipadamente e validados durante a compilação
(quando aplicável).
Essa diferença impacta diretamente a maneira como escrevemos, mantemos e depuramos código.

Python é uma linguagem interpretada e de tipagem dinâmica, o que significa que seu código não é compilado previamente
e que o tipo de uma variável é determinado em tempo de execução, sem exigir uma declaração explícita por parte do
programador.
Isso não significa que as variáveis não possuam um tipo associado, mas sim que o próprio interpretador do Python
gerencia automaticamente os tipos conforme necessário.

%Esse comportamento contrasta com linguagens de tipagem estática (como C, C++, Rust, etc.), onde a definição de tipos
%ocorre durante a compilação, antes do início da execução do programa.

Por exemplo, no seguinte código:
\begin{minted}{custompython}
descricao = 'Geladeira frost free 500l'  # tipo: str
quantidade = 10                          # tipo: int
peso_unitario = 72.7                     # tipo: float
\end{minted}

as variáveis \inlcode{descricao}, \inlcode{quantidade} e
\inlcode{peso\_unitario}\footnote{Em Python, o estilo \texttt{snake\_case} é tradicionalmente usado para nomear variáveis,
separando palavras com underscores (\texttt{\_}) e usando minúsculas.}
são automaticamente criadas com os tipos
\inlcode{str}, \inlcode{int} e \inlcode{float}, respectivamente.
O interpretador infere esses tipos com base nos valores atribuídos a cada variável
(o valor à direita de \texttt{=}).

Além disso, uma variável pode ser redefinida posteriormente, até mesmo para um tipo diferente.
Essa abordagem pode ser vantajosa para tornar o código mais flexível e dinâmico, mas também pode introduzir problemas ao
mascarar erros difíceis de identificar.

Por exemplo, em
\begin{minted}{custompython}
descricao = 'Geladeira frost free 500l'  # tipo: str
quantidade = 10                          # tipo: int
peso_unitario = 72.7                     # tipo: float

print(f"pré: 'descricao' é do tipo {type(descricao)} com valor: {descricao}")
descricao = quantidade   # redefine descricao para int
print(f"pós: 'descricao' é do tipo {type(descricao)} com valor: {descricao}")
\end{minted}
\begin{minted}{text}
pré: 'descricao' é do tipo <class 'str'> com valor: Geladeira frost free 500l
pós: 'descricao' é do tipo <class 'int'> com valor: 10
\end{minted}

a variável \inlcode{descricao} inicialmente criada como \inlcode{str} e valor \inlstr{Geladeira frost free 500l}, é redefinida para
\inlcode{int}, passando a armazenar o valor \inlcode{10}.
Esse comportamento exemplifica a tipagem dinâmica do Python, onde uma mesma variável pode assumir diferentes tipos ao
longo da execução do código.

Em Python, uma variável é apenas um nome que referencia um objeto em memória.
Quando uma variável é redefinida, ela passa a apontar para outro objeto, que não necessariamente precisa ter o mesmo tipo.

Esse código é válido e não resultará em erros de execução por si só.
No entanto, pode representar um erro lógico caso essa alteração de tipo não esteja alinhada com a intenção do programador.
Cabe ao programador garantir a coerência do
código, tratando \inlcode{descricao} como \inlcode{str} na parte inicial do programa e como
\inlcode{int} posteriormente, conforme necessário.

Embora essa redefinição com mudança de tipo não configure um erro sintático e possa ser útil em alguns casos, sua
prática indiscriminada é desencorajada, pois pode dificultar legibilidade e a manutenção do código.

A tipagem dinâmica do Python torna a linguagem mais simples e flexível, facilitando o desenvolvimento, especialmente
para iniciantes.
No entanto, essa característica exige atenção, pois erros de tipagem podem surgir em tempo de execução se os tipos
das variáveis não forem corretamente tratados.


Por exemplo, considere o seguinte código:
\begin{minted}{custompython}
descricao = 'Geladeira frost free 500l'  # tipo: str
quantidade = 10                          # tipo: int
peso_unitario = 72.7                     # tipo: float

peso_total = peso_unitario * quantidade
print(f'Peso total: {peso_total} kg')
\end{minted}
%
\begin{minted}{text}
    Peso total: 727.0 kg
\end{minted}

As variáveis \inlcode{quantidade} (\inlcode{int}) e \inlcode{peso\_unitario} (\inlcode{float}) são multiplicadas, gerando
um novo valor do tipo \inlcode{float}, que é atribuído a \inlcode{peso\_total}.
Essa operação é perfeitamente válida, tanto do ponto de vista sintático da linguagem quanto do ponto de vista lógico.

Agora, observe o que aconteceria se o programador, por descuido, digitasse \inlcode{descricao} ao invés
de \inlcode{peso\_unitario}.
\begin{minted}{custompython}
descricao = 'Geladeira frost free 500l'  # tipo: str
quantidade = 10                          # tipo: int
peso_unitario = 72.7                     # tipo: float

peso_total = peso_unitario * descricao   # <- erro de digitação
print(f'Peso total: {peso_total} kg')
\end{minted}
%
\begin{minted}[escapeinside=??]{text}
Traceback (most recent call last):
  File ?\textcolor{magenta!80}{\char34c:\char92curso\_python\char92aula1\char92variaveis.py\char34}?, line ?\textcolor{magenta!80}{5}?, in ?\textcolor{magenta!80}{<module>}?
    peso_total = ?\textcolor{red!70}{peso\_unitario * descricao}?
                 ?\textcolor{red!70}{\char126\char126\char126\char126\char126\char126\char126\char126\char126\char126\char126\char126\char126\char126\char94\char126\char126\char126\char126\char126\char126\char126\char126\char126\char126}?
?\textcolor{magenta!80}{TypeError: can't multiply sequence by non-int of type 'float'}?
\end{minted}

Durante a execução do código, o interpretador encontrou uma operação inválida e gerou um erro de incompatibilidade
de tipos, pois a multiplicação entre um \inlcode{float} com um \inlcode{str} não está definida.

Perceba que esse erro só ocorre em tempo de execução, no momento em que o interpretador Python tenta realizar a
multiplicação entre um \inlcode{float} e um \inlcode{str}.
Trata-se de um erro de lógica, não de sintaxe, pois, sendo Python uma linguagem de tipagem dinâmica,
\inlcode{peso\_unitario} e \inlcode{descricao} poderiam armazenar qualquer tipo de dado ao longo da execução do código,
podendo sofrer mudanças não só de valor, mas também de tipo, até atingir a linha onde ocorre a tentativa e
multiplicação e, consequentemente, o erro.

Esse erro seria facilmente detectado e corrigido no momento da escrita do código ou durante a compilação em
linguagens de tipagem estática (como C, C++, Java, Rust, etc), impedindo sua ocorrência em tempo de execução.
Esse exemplo foi incluído apenas para ilustrar também as desvantágens e os possíveis desafios da tipagem dinâmica.

Embora Python não suporte tipagem estática, ele suporta opcionalmente anotação de tipos.
As anotações de tipos são declarações do programador quanto ao tipo intentado para cada variável
criada\footnote{Além da anotação de tipo de variáveis, podemos também anotar tipos para parâmetros e retorno de
funções/métodos}.

Apesar das anotações de tipos não influenciarem a execução do código --- sendo ignoradas pelo interpretador Python
durante a execução --- elas são extremamente úteis.
Essas anotações são utilizadas por ferramentas de análise estática integradas à maioria das IDEs modernas para
identificar possíveis erros de incompatibilidade de tipos no momento da escrita do
código\footnote{A análise estática de tipos (ou \emph{type checking}) está habilitada por padrão na IDE PyCham.
Já no VS Code, ela pode habilitada incluindo os campos
\texttt{'python.analysis.autoImportCompletions': true,} \texttt{'python.analysis.typeCheckingMode': 'standard'} no arquivo de
configurações \texttt{settings.json}.
%Outras extensões populares para produtividade como \texttt{Pylint} e \texttt{Flake8} podem ser instaladas pelo VS Code Extensions Marketplace
}
além de melhorar a legibilidade e a manutenção, minimizando assim os riscos associados à tipagem dinâmica, mantendo
todos os seus benefícios.

Reescrevendo o exemplo anterior com anotação de tipos, a IDE agora pode nos alertar sobre um provável erro antes
mesmo da execução:
\begin{minted}[escapeinside=??]{custompython}
descricao: str = 'Geladeira frost free 500l'
quantidade: int = 10
peso_unitario: float = 72.7

peso_total = peso_unitario * descricao?\tikzmark{squigglydescricao}?   # <- erro de digitação
print(f'Peso total: {peso_total} kg')
\end{minted}
%
\begin{tikzpicture}[remember picture, overlay]
    \draw[orange, thick, line cap=round, decorate, decoration={snake, amplitude=0.15mm, segment length=1.5mm}]
        (pic cs:squigglydescricao) ++(-0.01,-0.03) -- ++(-3.66,0.0);
    \draw[->, thick, line cap=round, orange] (pic cs:squigglydescricao) ++(-0.01,-0.03) -- ++(0.6,0.6) node[right, fill=black!80!orange, inner sep=1pt]
        {\scriptsize\texttt{Operator '*'~not supported for types 'float'~and 'str'}};
\end{tikzpicture}

Além dos tipo primitivos, Python possui mais duas formas de armazenar informações: através de classes customizadas
\inlcode{class}, que serão discutidas no capítulo \ref{class} e através de coleções de dados (\inlcode{collections})




\section{Collections}

Em Python, coleções são estruturas que armazenam múltiplos valores em uma única variável.
As três coleções mais comuns da linguagem são: list, set e dict.
Cada uma delas possui características específicas que as tornam mais adequadas para diferentes situações.




\subsection{\inlcode{list}}
As listas (\inlcode{list}) são uma das estruturas mais versáteis do Python, representam sequências de elementos
ordenados e mutáveis.
Ous seja, os elementos inseridos nela são mantidos na mesma ordem e podem ser alterados a qualquer momento,
seja adicionando, removendo modificando ou substituindo itens.

Além disso, possuem tamanho arbitrário, crescendo conforme novos elementos são inseridos.

Uma lista pode ser criada utilizando colchetes \inlcode{[]} e pode armazenar qualquer tipo de dado, inclusive, diferentes tipos para cada elemento.
Por exemplo:
\begin{minted}{custompython}
numeros = [8, 2, 0, -4, 15]
nomes = ['Alice', 'Bruno', 'Carla']
misto = [42, 'texto', 3.14, True]  # lista com tipos variados
\end{minted}

A lista é uma coleção dinâmica, e pode crescer mesmo apos sua criação para acomodar novos elementos:
\begin{minted}{custompython}
cores = ['azul', 'amarelo', 'cinza', 'vermelho']
print(f'pré: A lista 'cores' possui {len(cores)} elementos: {cores}')

cores.append('verde')  # adiciona um novo elemento de valor 'verde' ao final da lista
print(f'pós: A lista 'cores' possui {len(cores)} elementos: {cores}}')
\end{minted}
\begin{minted}{text}
pré: A lista 'cores' possui 4 elementos: ['azul', 'amarelo', 'cinza', 'vermelho']
pós: A lista 'cores' possui 5 elementos: ['azul', 'amarelo', 'cinza', 'vermelho', 'verde']
\end{minted}

Esse código introduz um elemento novo: a chamada cores.append(), onde append() é um método da método da classe list
Métodos são semelhantes a funções, mas estão associados a um tipo de dado específico (à uma \inlcode{class}) e
normalmente modificam o próprio objeto ao qual pertencem.

A chamada de um método geralmente impacta diretamente o objeto no qual ele é invocado.
No caso de append(), um novo elemento é adicionado ao final da lista, alterando sua estrutura inicial.

Esse conceito será revisitado e formalizado com mais detalhes no Capítulo \ref{class}, quando exploraremos a
construção e uso de classes em Python.

Note que a lista cresce automaticamente para acomodar novos elementos, sem que o programador precise gerenciar ou
alocar memória adicional manualmente.
Essa característica torna as listas particularmente flexíveis para armazenar conjuntos dinâmicos de dados.

Além disso, no último exemplo, introduzimos a função \inlcode{len()}, que retorna o número de elementos presentes na coleção.
Essa é uma função essencial para validar o tamanho de uma lista antes de realizar operações que dependam da quantidade de itens.

Também podemo acessar elementos individuais dessa lista, tanto para leitura como para escrita.
Essa indexação é realizada com a seguinte sintaxe \inlcode{nome_da_lista[indice]}, veja o código exemplo:
\begin{minted}{custompython}
cores = ['azul', 'amarelo', 'cinza', 'vermelho', 'verde']

# a indexação dos elemntos da listas por um inteiro inicia em 0
selecionada1 = cores[1]  # indice 1 equivale ao 2° elemento
print(f'selecioanda1 é: {selecionada1}')

# indice negativos podem ser usados para indexar elemntos do final para o início da listas
selecionada2 = cores[-2]  # -2 indica o penúltimo elemento
print(f'selecioanda2 é: {selecionada2}')

# é possível a indexação por slice, ou faixa de índices m:n (n é não inclusivo)
selecionadas = cores[0:3]  # 0:3 corresponde aos indices 0,1,2
print(f'selecionadas são: {selecionadas}')

# elementos da lista também podem ser individualemente modificados
cores[2] = 'marrom'  # altera o terceiro elemento de 'cinza' para 'marrom'
print(f'pós modificação: {cores}')

# podemo excluir elementos, o que desloca dos indices dos elementos a sua direita
del cores[3]
print(f'pós excusão por indice: {cores}')

# ou ainda podemo excluir não por indice mas por valor
cores.remove('azul')
print(f'pós excusão por valor: {cores}')

# tentar acessar índices que não existe resultará em um erro
cores[99]
\end{minted}
\begin{minted}[escapeinside=??]{text}
selecioanda1 é: amarelo
selecioanda2 é: vermelho
selecionadas são: ['azul', 'amarelo', 'cinza']
pós modificação: ['azul', 'amarelo', 'marrom', 'vermelho', 'verde']
pós excusão por índice: ['azul', 'amarelo', 'marrom', 'verde']
pós excusão por valor: ['amarelo', 'marrom', 'verde']
Traceback (most recent call last):
  File ?\textcolor{magenta!80}{\char34c:\char92curso\_python\char92aula1\char92variaveis.py\char34}?, line ?\textcolor{magenta!80}{28}?, in ?\textcolor{magenta!80}{<module>}?
    ?\textcolor{red!70}{cores[99]}?
    ?\textcolor{red!70}{\char126\char126\char126\char126\char126\char94\char94\char94\char94}?
?\textcolor{magenta!80}{IndexError: list index out of range}?
\end{minted}

Vários outros métodos e funções para manipular listas estão disponíveis por padrão. Segue uma breve enumeração dos mais úteis:
\begin{itemize}
\item \inlcode{len(lista)} – Retorna o número de elementos na lista.
\item \inlcode{lista.append(valor)} – Adiciona um novo elemento ao final da lista.
\item \inlcode{lista.insert(indice, valor)} – Insere um elemento em uma posição específica.
\item \inlcode{lista.remove(valor)} – Remove a primeira ocorrência do valor especificado.
\item \inlcode{lista.pop(indice)} – Remove e retorna o elemento na posição indicada (ou o último, se omitido).
\item \inlcode{lista.index(valor)} – Retorna o índice da primeira ocorrência do valor.
\item \inlcode{lista.count(valor)} – Retorna o número de vezes que o valor aparece na lista.
\item \inlcode{lista.sort()} – Ordena a lista em ordem crescente (por padrão).
\item \inlcode{lista.reverse()} – Inverte a ordem dos elementos da lista.
\item \inlcode{sorted(lista)} – Retorna uma nova lista ordenada, sem modificar a original.
\item \inlcode{lista.copy()} – Retorna uma cópia superficial da lista.
\item \inlcode{lista.clear()} – Remove todos os elementos da lista.
\end{itemize}

Esses métodos permitem desde operações simples, como acessar elementos, até manipulações mais avançadas,
como ordenação e remoção de elementos.
No decorrer desse tutorial, abordaremos mais detalhes sobre a aplicação desses métodos em exemplos práticos.


\subsection{\inlcode{tuple}}

Assim como \inlcode{list}, a \inlcode{tuple} (ou tupla) é uma estrutura de dados que permite armazenar uma sequência
de valores.
No entanto, enquanto listas são mutáveis (ou seja, seus elementos podem ser alterados, adicionados ou removidos),
tuplas são imutáveis: uma vez criadas, seus elementos não podem ser modificados.

Essa característica confere à tupla maior segurança e previsibilidade, tornando-a ideal para representar coleções
fixas de dados que não devem ser alteradas acidentalmente, ou até valores de retorno de uma função.

Tuplas são definidas por uma sequência de valores separados por vírgula (\inlcode{,}), não necessariamente, mas comumente
inseridos entre parênteses (\inlcode{()}).
\begin{minted}{custompython}
coordenada = (46.8, 22.3)
print(coordenada)
\end{minted}
\begin{minted}[escapeinside=??]{text}
(46.8, 22.3)
\end{minted}

Para definir uma tupla de um único elemento, é obrigatório adicionar uma vírgula,
caso contrário, o Python interpretará o valor como um tipo isolado:
\begin{minted}{custompython}
numero = (5)   # não é tupla, é apenas um int
tupla = (5,)  # tupla com um único elemento
print(f'numero é do tipo {type(numero)} e valor: {numero}')
print(f'tupla é do tipo {type(tupla)} e valor: {tupla}')
\end{minted}
\begin{minted}[escapeinside=??]{text}
numero é do tipo <class 'int'> e valor: 5
tupla é do tipo <class 'tuple'> e valor: (5,)
\end{minted}


A indexação de tuplas segue o mesmo princípio das listas,
permitindo acessar elementos por índices (\inlcode{0} para o primeiro item, índices negativos para contar do final)
\begin{minted}{custompython}
cores = 'azul', 'verde', 'vermelho', 'amarelo', 'roxo'
print(cores[0])     # azul
print(cores[-1])    # roxo
print(cores[1:4])   # ('verde', 'vermelho', 'amarelo')
\end{minted}
\begin{minted}[escapeinside=??]{text}
azul
roxo
('verde', 'vermelho', 'amarelo')
\end{minted}

Embora tupla possa pareceu uma versão limitada de lista, elas são estruturas complementares, com propósitos distintos.
Essas limitações justamente tornam operações com tuplas mais previsíveis, rápidas e seguras.




\subsection{\inlcode{set}}
Um set é uma coleção semelhante à lista (list), porém não permite valores duplicados e não mantém uma ordem fixa dos
elementos.
Isso faz com que seja uma estrutura útil para armazenar itens únicos e realizar operações matemáticas como união,
interseção e diferença.

Os conjuntos podem ser criados utilizando chaves \inlcode{\{\}} ou a função \inlcode{set()}:
\begin{minted}{custompython}
cores = {'azul', 'vermelho', 'verde', 'azul'}
print(cores)
\end{minted}
\begin{minted}[escapeinside=??]{text}
{'azul', 'vermelho', 'verde'}
\end{minted}

Observe que, mesmo que \inlstr{azul} tenha sido declarado duas vezes, ele aparece apenas uma vez no conjunto.

Outra forma de criar um conjunto é utilizando \inlcode{set()} passando uma \inlcode{list} como argumento:
\begin{minted}{custompython}
valores = set([1, 2, 3, 4, 4, 5])
print(valores)
\end{minted}
\begin{minted}[escapeinside=??]{text}
{1, 2, 3, 4, 5}
\end{minted}

Os conjuntos são ideais quando precisamos garantir que os elementos sejam únicos, eliminando valores repetidos automaticamente.

Por não manter uma ordem fixa dos elementos, set não permite acesso por índice.
Isso significa que não podemos utilizar a notação \inlcode{set[indice]} para recuperar ou modificar elementos
específicos, como fazemos com listas.
Em vez disso, a manipulação de conjuntos é realizada por métodos próprios, como adição, remoção, união, interseção e
diferença, que operam sobre o conjunto como um todo:

\begin{minted}{custompython}
a = {1, 2, 3, 4}
b = {3, 4, 5}
b.add(6)

print(f'união: {a | b}')
print(f'interseção: {a & b}')
print(f'diferença: {a - b}')
\end{minted}
\begin{minted}[escapeinside=??]{text}
união: {1, 2, 3, 4, 5, 6}
interseção: {3, 4}
diferença: {1, 2}
\end{minted}

Embora \inlcode{set} não suporte indexação direta, ainda é possível iterar sobre seus elementos utilizando laços
como \inlcode{for}.
Dessa forma, podemos percorrer todos os itens do conjunto sem precisar acessar um elemento
específico por índice.
A iteração sobre coleções de dados será abordada no capítulo \ref{iffor}.

Aqui está uma lista resumida dos principais métodos e funções para manipulação de sets em Python:
\begin{itemize}
\item \inlcode{len(set)} – Retorna o número de elementos no conjunto.
\item \inlcode{set.add(valor)} – Adiciona um novo elemento ao conjunto.
\item \inlcode{set.remove(valor)} – Remove um elemento existente (gera erro se o valor não existir).
\item \inlcode{set.discard(valor)} – Remove um elemento sem gerar erro caso ele não exista.
\item \inlcode{set.pop()} – Remove e retorna um elemento aleatório do conjunto.
\item \inlcode{set.clear()} – Remove todos os elementos do conjunto.
\item \inlcode{set.copy()} – Retorna uma cópia do conjunto.
\end{itemize}


\subsection{\inlcode{dict}}

Em Python, um dicionário (dict) é uma estrutura de dados que implementa um hashmap, permitindo o armazenamento de pares
\inlcode{key: value} (chave e valor).
Essa abordagem garante acesso eficiente aos valores a partir de suas chaves, tornando a busca em grandes coleções
extremamente rápida.

Diferente das listas \inlcode{list} e conjuntos \inlcode{set}, que organizam elementos sequencialmente ou de forma desordenada,
o dicionário associa cada valor a uma chave única, permitindo acessos diretos sem necessidade de percorrer toda a
estrutura.

Cada chave (\inlcode{key}) pode ser qualquer tipo de dado que suporte a função \inlcode{hash()}, como números, strings
ou tuplas imutáveis.
Já o valor (\inlcode{value}) pode ser qualquer objeto do Python, incluindo listas, outras coleções, classes definidas
pelo usuário e até mesmo funções.

Podemos criar um dicionário utilizando chaves (\inlcode{\{\}}) e inserindo os pares \inlcode{key: value} dentro delas.
Por exemplo:
\begin{minted}{custompython}
dados = {'nome': 'Fulano', 'idade': 25, 'cidade': 'Florianópolis'}
print(dados)
\end{minted}
\begin{minted}[escapeinside=??]{text}
{'nome': 'Fulano', 'idade': 25, 'cidade': 'Florianópolis'}
\end{minted}

Para recuperar um valor armazenado em um dicionário, utilizamos sua chave entre colchetes \inlcode{[]}:
\begin{minted}{custompython}
dados = {'nome': 'Fulano', 'idade': 25, 'cidade': 'Florianópolis'}
nome = dados['nome']
print(f'nome: {nome}')
\end{minted}
\begin{minted}[escapeinside=??]{text}
nome: Fulano
\end{minted}

Tentar acessar uma chave inexistente gera um erro \inlcode{KeyError}:
\begin{minted}{custompython}
dados = {'nome': 'Fulano', 'idade': 25, 'cidade': 'Florianópolis'}
telefone = dados['telefone']
\end{minted}
\begin{minted}[escapeinside=??]{text}
Traceback (most recent call last):
  File ?\textcolor{magenta!80}{\char34c:\char92curso\_python\char92aula1\char92variaveis.py\char34}?, line ?\textcolor{magenta!80}{2}?, in ?\textcolor{magenta!80}{<module>}?
    ?\textcolor{red!70}{telefone = dados['telefone']}?
    ?\textcolor{red!70}{           \char126\char126\char126\char126\char126\char94\char94\char94\char94\char94\char94\char94\char94\char94\char94\char94\char94}?
?\textcolor{magenta!80}{KeyError: 'telefone'}?
\end{minted}

Já escrever em uma chave previamente inexistes, expande o \inlcode{dict} incluindo esse novo par:
\begin{minted}{custompython}
dados = {'nome': 'Fulano', 'idade': 25, 'cidade': 'Florianópolis'}

dados['telefone'] = '(48)99999-9999'
print(dados)
\end{minted}
\begin{minted}[escapeinside=??]{text}
{'nome': 'Fulano', 'idade': 25, 'cidade': 'Florianópolis', 'telefone': '(48)99999-9999'}
\end{minted}

Dicionários são mutáveis, permitindo que seus elementos sejam alterados, adicionados ou removidos.
Podemos modificar ou remover um valor acessando sua chave diretamente:
\begin{minted}{custompython}
dados = {'nome': 'Fulano', 'idade': 25, 'cidade': 'Florianópolis'}

dados['idade'] = 26   # modifica o valor associado à chave 'idade'
del dados['cidade']   # remove o valor associado à chave 'cidade'
print(dados)
\end{minted}
\begin{minted}[escapeinside=??]{text}
{'nome': 'Fulano', 'idade': 26}
\end{minted}

Métodos úteis para manipulação de dicionários


Aqui estão alguns métodos úteis para trabalhar com dicionários:
\begin{itemize}
\item \inlcode{len(dicionario)}: Retorna a quantidade de pares chave-valor no dicionário.
\item \inlcode{dicionario.keys()}: Retorna todas as chaves do dicionário.
\item \inlcode{dicionario.values()}: Retorna todos os valores do dicionário.
\item \inlcode{dicionario.items()}: Retorna todos os pares chave-valor como tuplas.
\item \inlcode{dicionario.pop(chave)}: Remove um item e retorna seu valor.
\item \inlcode{dicionario.update(outro_dicionario)}: Atualiza o dicionário com novos pares chave-valor.
\item \inlcode{dicionario.clear()}: Remove todos os elementos do dicionário.
\end{itemize}

Dicionários são ferramentas poderosas para armazenar e acessar dados de forma eficiente.
Eles são amplamente utilizados em Python para estruturar informações e otimizar operações de busca.


