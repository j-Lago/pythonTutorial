Escreva uma função para validar usuário e senha.
Ao ser executado, o programa solicita ao usuário (via terminal) seu login e senha.
Os valores são então comparados a um cadastro pré-definido (armazenado em um dicionário global).
A função retorna \inlcode{True} ou \inlcode{False}, além de exibir uma mensagem no terminal informando se o acesso foi concedido ou negado.
\begin{minted}{custompython}
def validacao(usuario: str, senha: str) -> bool:
    global CADASTROS
    # seu código aqui


# logins e senhas previamente cadastrados
CADASTROS = {
    'fulano'  : '8g7&4A5',
    'beltrano': 'Hgr787@',
    'sicrano' : 'po78BA-'
    }

# interrompe a execução e solicita usuário e senha
usuario = input("Digite seu login:")
senha = input("Digite sua senha:")

# chama a função de validação
acesso = validacao(usuario, senha)
\end{minted}

\subsection*{Desafio:}
Modifique o programa para permitir ao usuário três tentativas de acesso.
Caso a senha seja digitada incorretamente três vezes consecutivas, essa senha deve ser permanentemente invalidada.
Crie uma nova mensagem informando essa situação.
A invalidação da senha de um usuário não deve afetar os demais.


\subsection*{Importante:}
Tenha em mente que este exercício tem como objetivo explorar conceitos introdutórios de comparação e controle de fluxo.
No entanto, por razões óbvias de segurança, senhas nunca devem ser armazenadas ou comparadas diretamente.
Em aplicações reais, a validação de credenciais sempre envolve criptografia robusta e bancos de dados especializados,
projetados para garantir a proteção das informações.




