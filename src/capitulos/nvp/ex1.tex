\chapter{Exercícios}

O desenvolvimento do raciocínio lógico e sua aplicação na programação são habilidades que se aprimoram, sobretudo, pela
prática.
Por isso, antes de explorarmos novos recursos, é essencial consolidarmos o que já estudamos.

Até agora, apresentamos pequenos trechos de código para ilustrar os principais mecanismos da linguagem.
Python oferece uma ampla gama de funcionalidades, mas os conceitos abordados até aqui — variáveis, tipos, expressões,
funções e controle de fluxo — são fundamentais em qualquer linguagem de programação e já possibilitam a construção de
diversas soluções.

Os problemas propostos neste capítulo, embora em alguns casos se inspirem em desafios práticos reais, não têm como
objetivo desenvolver soluções otimizadas para produção.
Em vez disso, buscam fortalecer o raciocínio lógico, a capacidade de resolver problemas e a habilidade de transcrever
soluções para o computador.
Além disso, buscam desenvolver uma maior familiaridade com a linguagem, a IDE e o ecossistema que a envolve.

\newcommand{\questao}[2][]{\addtocounter{section}{1}\section*{Questão \thesection} \input{#1#2}}

\newcommand{\expath}{capitulos/nvp/ex/}

\questao[\expath]{desvio_padrao}
\questao[\expath]{login}
\questao[\expath]{separa_pares_impares}
\questao[\expath]{seno}
\questao[\expath]{produto_mais_vendido}
\questao[\expath]{alunos_aprovados}
\questao[\expath]{media_movel}
\questao[\expath]{ordena}
\questao[\expath]{procura}
