%! Author = j-Lago
%! Date = 6/28/2024

\unnumberedchapter{Orientações Iniciais e Convenções}

Este tutorial oferece uma introdução concisa ao Python, voltada para os alunos ingressantes no curso de Mestrado
Profissional em Sistemas de Energia.

Por ter um caráter introdutório, o texto apresenta exemplos e conceitos sem a intenção de esgotar completamente o
tema ou fornecer soluções otimizadas para problemas específicos.

Os exemplo aqui descritos assumem a utilização de Python 3.10 (ou superior) que deve estar devidamente instalado (\url{https://www.python.org/downloads/}).

Também recomenda-se a utilização de alguma IDE (\emph{Integrated Development Environment}) para facilitar a escrita e manipulação do código.
As duas mais populares para desenvolvimento em Python são:
\begin{itemize}
    \item \texttt{PyCharm}: uma IDE dedicada ao Python, desenvolvida pela JetBrains.
    Oferece uma configuração padrão robusta e inclui funcionalidades como autocomplete, análise de código, refatoração,
    debugging, testes integrados e integração com Git.
    É uma ferramenta profissional ideal para projetos complexos, embora possa ser pesada em máquinas mais simples ou antigas.
    Disponível em duas versões: Professional (paga) e Community Edition (gratuita).
    Embora a versão gratuita tenha algumas limitações, ela é mais do que suficiente para este tutorial e até mesmo para projetos pessoais de grande porte.
    (\url{https://www.jetbrains.com/pycharm/download/})
    \item \texttt{VS Code}: um editor de código aberto, leve e flexível, não restrito a uma única linguagem.
    O suporte ao Python é obtido por meio de extensões, que também habilitam ferramentas de análise de código.
    Seu uso exige uma configuração inicial, incluindo a instalação da extensão Python (Microsoft).
    Outras extensões relacionadas à produtividade podem ser obtidas pelo marketplace integrado ao VS Code.
    (\url{https://code.visualstudio.com/download/})
\end{itemize}


No texto, códigos Python serão identificados por uma caixa preta:
\begin{minted}[escapeinside=??, frame=single, rulecolor=brown]{custompython}
?\textcolor{brown}{\faIcon{file} exemplo\_codigo\_python.py}?
\end{minted}
\vspace{-0.8em}
%
\begin{minted}[escapeinside=??]{custompython}

print('Hello, World!')
\end{minted}

enquanto comandos e mensagens impressas no terminal estarão em uma caixa azul:
\begin{minted}[escapeinside=??]{text}
?\textcolor{green!20!brown}{C:\char92curso\_python\char92>}? python exemplo_codigo_python.py
Hello, World!
\end{minted}